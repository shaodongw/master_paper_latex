\BiChapter{数学公式的输入方法}{Input methods of equations}

\BiSection{行内公式}{Inline mode equations}

出现在正文一行之内的公式称为行内公式,例如~$f(x)=\int_{a}^{b}\frac{\sin{x}}{x}\mathrm{d}x$。对于非矩阵和非多行形式的行内公式,一般不会使得行距发生变化。

\BiSection{行间公式}{Displaymath mode equations}

位于两行之间的公式称为行间公式,每个公式都是一个单独的段落,下边的例子是一个无编号的行间单行公式

\[
\int_a^b{f\left(x\right)\mathrm{d}x}=\lim_{\left\|\Delta{x_i}\right\|\to 0}\sum_i{f\left(\xi_i\right)\Delta{x_i}}
\]

下边的例子是一个无编号的行间多行公式(\ref{lizi})
\begin{eqnarray*}\label{lizi}
\sin 2x&=&2\sin x\cos x\\
\cos 2x&=&2\cos x^2-1=1-2\sin x^2=\cos x^2-\sin x^2
\end{eqnarray*}